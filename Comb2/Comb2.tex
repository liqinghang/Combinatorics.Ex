\documentclass[UTF8]{ctexart}
\usepackage{algorithm}
\usepackage{algorithmic}
\usepackage{amsmath,amssymb}
\renewcommand{\algorithmicrequire}{ \textbf{Input:}} %Use Input in the format of Algorithm
\renewcommand{\algorithmicensure}{ \textbf{Output:}} %UseOutput in the format of Algorithm
\begin{document}
李青航 SA22225226\\
\textbf{15}\\
\begin{equation}
\nonumber
\begin{split}
(\neg p\wedge(p\rightarrow q))\rightarrow \neg q &\equiv \neg(\neg p\wedge(p\rightarrow q))\vee \neg q\\
& \equiv \neg(\neg p \wedge (\neg p \vee q))\vee \neg q\\
& \equiv \neg(\neg p )\vee \neg q\\
& \equiv p \vee \neg q
\end{split}
\end{equation}
不是重言式\\

\noindent\textbf{16}\\
\begin{equation}
\nonumber
\begin{split}
(\neg q \wedge (p \rightarrow q))\rightarrow \neg p
& \equiv \neg(\neg q \wedge(\neg p \vee q)) \vee \neg p\\
& \equiv (q \vee \neg (\neg p \vee q)) \vee \neg p\\
& \equiv (q \vee (p \wedge \neg q))\vee \neg p\\
& \equiv ((q \vee p) \wedge (q \vee \neg q))\vee \neg p \\
& \equiv ((p\vee q) \wedge T)) \vee \neg p\\
& \equiv q \vee p\vee \neg p \\
&\equiv T
\end{split}
\end{equation}
是重言式\\

\noindent\textbf{20}

\textbf{i.}Alice is a mathematics major. Therefore, Alice is either a mathematics major or a computer science major.

\textbf{p:}Alice is a mathematics major.

\textbf{q:}Alice is a computer science major.

rule of inference:$p\Rightarrow (p\vee q)$\\

\textbf{ii.}Jerry is a mathematics major and a computer science major. Therefore, Jerry is a mathematics major.

\textbf{p:} Jerry is a mathematics major;

\textbf{q:} Jerry is a computer science major.

rule of inference: $(p\vee q)\Rightarrow p$\\

\textbf{iii.}If it is rainy, then the pool will be closed. It is rainy. Therefore, the pool is closed.

\textbf{p:} It is rainy;

\textbf{q:} The pool is closed.

rule of inference: $(p \rightarrow q) \wedge p \Rightarrow q$\\

\textbf{iv.}If it snows today, the university will close. The university is not closed today. Therefore, it did not snow today.

\textbf{p:}It snows today;

\textbf{q:}The university is closed today.

rule of inference:$(p\rightarrow q)\wedge \neg q \Rightarrow \neg q$\\

\textbf{v.}If I go swimming, then I will stay in the sun too long. If I stay in the sun too long, then I will sunburn. Therefore, If I go swimming, then I will sunburn.

\textbf{p:} I go swimming;

\textbf{q:} I stay in the sun too long;

\textbf{r:} I sunburn.

rule of inference:$(p\rightarrow q)\wedge (q\rightarrow r)\Rightarrow (q\rightarrow r)$\\

\noindent\textbf{22}\\

\textbf{i.}$A\cup(B\cup C)=(A\cup B) \cup C$ 显然,这是成立的,证明略

\textbf{ii.}$A\cap(B\cap C)=(A\cap B) \cap C$ 显然,这是成立的,证明略

\textbf{iii.}
\begin{equation}
\nonumber
\begin{split}
x\in A\cup (B\cap C) 
&\Leftrightarrow x\in A \mbox{ or }x \in (B \cap C)\\
&\Leftrightarrow (x \in A \mbox{ or } x\in B) \mbox{ and } (x \in A  \mbox{ or }  x\in C)\\
&\Leftrightarrow x\in (A\cup B \mbox{ and }A \cup C)\\
&\Leftrightarrow x \in (A\cup B) \cap (A \cup C)
\end{split}
\end{equation}

\noindent\textbf{24}\\

整数域内,是否是满射?

\textbf{i.}$f(m,n)=m+n$ 是

\textbf{ii.}$f(m,n)=m^2+n^2$ 不是

\textbf{iii.}$f(m,n)=m$ 是

\textbf{iv.}$f(m,n)=|n|$ 不是

\textbf{v.}$f(m,n)=m-n$ 是


\noindent\textbf{26}\\

$\sum_{k=1}^{5}(k+1)=20$

$\sum_{j=0}^{4}(-2)^j=11$

$\sum_{i=1}^{10}3=30$

$\sum_{=0}^{8}(2^{j+1}-2^{j})=511$

\end{document}