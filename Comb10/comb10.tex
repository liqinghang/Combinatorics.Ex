\documentclass[UTF8]{ctexart}

\usepackage{algorithm}
\usepackage{algorithmic}
\usepackage{amsmath,amssymb}
\usepackage{booktabs}
\usepackage{geometry}
\usepackage{tikz}
\usepackage{color}
\geometry{a4paper,scale=0.7}

\begin{document}
    SA22225226 李青航

    \noindent\textbf{354}

    组合意义:把 $p$ 个元素分为 $k$ 组(非空),共有 $S(p,k)$ 种分法

    \textbf{i. }
    
    $S(n, 1) = 1,~~~~(n \ge 1)$

    把$n$个元素分为非空的1组,只有一种分法

    \textbf{ii. }
    
    $S(n, 2) = 2^{n-1} - 1,~~~~(n \ge 2)$

    把$n$个元素,分为非空的两组。
    先选出一部分作为一组,剩下作为另外一组。

    先选一部分方法数$\sum^{n}_{k=0}{n\choose k}=2^n$ ;
    选一部分和选他的补集一样,所以再除以2;再减去非空选0个这种情况,减1

    \textbf{iii. }

    $S(n, n - 1) = {n\choose 2}~~~~, (n\ge 1)$

    有且仅有2个人一组,其他是1个人一组,先选2个人在一组
    方法数 ${n\choose 2}$ 剩余的自动一人一组了

    \textbf{iv. }

    $S(n, n - 2) = {n\choose 3}+3{n\choose 4},~~~~(n\ge 2)$

    情况1,分组元素数为$3, 1, 1, 1, 1\dots$
    共$n \choose 3$种

    情况2,分组元素数为$2,2,1,1,1\dots$
    先选出4个元素$n\choose 4$种,4个元素可分为${4\choose 2} \div 2$=3种

    ~\\
    \noindent\textbf{369}

    反证:
    
    如果自共轭分拆数,当$n\ge 2$只有1部分,
    Frrers图就只有1行(或1列),关于对角线翻转后,变为1列(1行)
    就不关于对角线对称

    ~\\
    \noindent\textbf{374}

    $\sqrt{x^2-6x+1}$的麦克劳林 展开$1-3x-4x^2-12x^3-44x^4+\cdots$

    \begin{equation*}
        \begin{aligned}
            \sum_{k=0}^{\infty} R_n x^n =& \frac{1}{2x} (-(x-1) - (1-3x-4x^2-12x^3-44x^4 + \cdots)) \\
            =& \frac{2x+4x^2+ 12x^3 + 44x^4 + \cdots}{2x} \\
            =& 1 + 2x + 6x^2 + 22 x^3 + \cdots
            \end{aligned}
    \end{equation*}

    综上,有$R_0 = 1, R_1 = 2, R_2 = 6, R_3 = 22\dots$

    \noindent\textbf{378}

    $X=\{x_1,x_2,\dots x_6\},~~Y=\{y_1,y_2,\dots y_6\}$

    \usetikzlibrary {graphs.standard}

    \begin{tikzpicture}
        \graph {
      subgraph I_nm [n=6, m=6];
    
      V 1 -- { W 1, W 2, W 3, W 6 };
      V 2 -- { W 2, W 3, W 5, W 6};
      V 3 -- { W 2, W 3, W 4, W 5 };
      V 4 -- { W 6 };
      V 5 -- { W 4 , W 5 , W 6};
      V 6 -- { W 1, W 2,W 5,W 6};
    };
    \end{tikzpicture}

    一个匹配\{1,1\};\{2,2\};\{3,3\};\{4,6\};\{5,4\};\{6,5\}

    \noindent\textbf{379}

    % \usepackage{booktabs}


\begin{table}[h]
    \centering
    \begin{tabular}{|c|c|c|c|c|c|} 
    \toprule
    $b_1$ & $w_1$ & $b_2$& x & x & $w_2$  \\ 
    \hline
    x & $b_3$ & $w_4$ & x & $w_5$ & $b_4$  \\ 
    \hline
    x & $w_6$ & $b_5$ & $w_7$ & $b_6$ & x  \\ 
    \hline
    x & x & x & x & x & $b_7$  \\ 
    \hline
    x & x & x &$ w_8$ & $b_8$ & $w_9$  \\ 
    \hline
    $b_9$ & $w_{10}$ & x & x & $w_3$ & $b_{10}$  \\
    \bottomrule
    \end{tabular}
    \end{table}

    线太乱,略
    \begin{tikzpicture}
        \graph {
      subgraph I_nm [n=10, m=10];

    };
    \end{tikzpicture}

    一个覆盖是 $b_1w_1,b_3w_6,b_2w_4,b_5w_7\dots$总之存在

    \noindent\textbf{378}

    就是二分图的最大匹配,最大匹配7
\end{document}