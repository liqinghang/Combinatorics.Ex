\documentclass[UTF8]{ctexart}
\usepackage{algorithm}
\usepackage{algorithmic}
\usepackage{amsmath,amssymb}
\usepackage{booktabs}


\renewcommand{\algorithmicrequire}{ \textbf{Input:}} %Use Input in the format of Algorithm
\renewcommand{\algorithmicensure}{ \textbf{Output:}} %UseOutput in the format of Algorithm
% 参考:https://blog.csdn.net/jzwong/article/details/52399112

\begin{document}

SA22225226 李青航

\noindent \textbf{74}

将分数记为两个正整数$m$和$n$的比值,即$m/n$。对于整数$i = 0,1, \cdots , n$,考察分数$10^i m / n$,并且记余数为$r_i$,显然余数的取值范围是$r_i = 0, 1, \cdots, n-1$,一共有$n$个,因此由鸽巢原理可以断定,存在整数$i, j (0 \le i < j \le n)$满足$r_i = \ r_j$

之后考虑分数$(10^jm-10^im) / n$,并且记$s = j-i$,这样存在整数$q$满足$10^{i}(10^s - 1)m = nq$,并且记$q/(10^s-1)$的余数为$r$。同样可以判断$r$的取值范围是$r = 0, 1, \cdots, 10^s -2$。可以写作$q = b(10^s -1) + r$

那么分数$10^i m /n$就可以展开为等比级数的和
\begin{equation}
\nonumber
\begin{split}
\frac{10^i m}{n}&=\frac{q}{10^s -1 }\\
&=b+\frac{r}{10^s-1}\\
&=b+ \frac{r}{10^s}+\frac{r}{10^{2s}}+\frac{r}{10^{3s}}+...+\frac{r}{10^{ns}}+....
\end{split}
\end{equation}

所以,$10^i m/n$可以表示为循环小数的形式,循环部分的长度是$j-i$,那么$m/n$是$10^im/n$小数点左移i位,不改变循环部分,因此最终也是循环的。

~\\

\noindent \textbf{85}    \underline{看错了,这题没布置,多做了}

\textbf{i. }将边长为1的正三角形,划分为4个相同的边长为$\frac{1}{2}$的小正三角形,第1--4个点分别在4个小正三角形中,根据鸽巢原理,存在第5个点,与其中1个点在同一个小正三角形中,两点最大距离为$\frac{1}{2}$ (分别在两个顶点)

\textbf{ii. }同理,划分为边长为$\frac{1}{3}$的9个正三角形

\textbf{iii. }将其划分为$m_n-1=n^2$个边长为$\frac{1}{n}$的小正三角形,在大三角形中,有$m_n$个点,根据鸽巢原理,存在两个点在同一个小三角形中,其最大距离为$\frac{1}{n}$(在三角形的两个顶点)

~\\

\noindent \textbf{86}

考虑$K_{17} \rightarrow K_3, K_3, K_3$,并且使用红色、蓝色和绿色进行染色。

任选一个点x,与它相连的边有16条,由鸽巢原理加强版可知,至少存在$\lceil \dfrac{16}{3} \rceil = 6$条边的颜色相同,不妨设为红色。

那么对于与$x$相连的点$\{ x\} _{i=1}^6$,如果他们之中的连线存在一条红边,则该条边上的两点和$x$则可以构成一个$K_3$;否则(即不存在红边),由$r(3, 3) = 6$可知,6个顶点染色一定可以构造出一个蓝色或绿色的$K_3$。

综上,17个点可以构造出$K_3$,不少于17个点也一定能构造出$K_3$,即$r(3,3,3) \le 17$

~\\
\noindent \textbf{95}

没有限制:四位数,每一位都是5种取法,共$5^4$种

(a)限制:$5\times 4 \times 3 \times 2$种

(b)限制:末尾是偶数,末尾2种取法,其他三位都是5种取法,共$2\times 5^3$种

(a)(b)限制:末尾是偶数2种取法,其他三位各,4,3,2种取法,共$2\times 4\times 3 \times 2$种

~\\
\noindent \textbf{101}

\textbf{i. }葫芦\\
先选一对,再选三条;或者先选三条,再选一对\\
共$13\times {4\choose 2}\times 12 \times {4\choose 3}$种\\

\textbf{ii. }顺子\\
顺子从12345到10JQKA有10种,其中每张牌花色都有四种可能,减去40种同花顺\\
共$10\times 4^5-40$种\\

\textbf{iii. }同花\\
4种花色,每种13张牌选5张,减去40种同花顺\\
共$4\times {13\choose 5}-40$种\\

\textbf{iv. }同花顺\\
顺子从12345到10JQKA有10种,因为要同花色,所以只乘4,共$4\times 10=40$种\\


\textbf{v. }两对\\
第一步选数字,两对中的数字组合,13个数字选2个数字,有${13 \choose 2}$种\\
第二步,第一对牌又是4种花色种选2种,有${4\choose 2}$\\
第三步,第二对牌同理\\
第四步,第5张牌,不能和两对中的数字一样(不然就变葫芦了)$52-8=44$
所以,一共${13\choose 2}\times {4\choose 2}^2\times 44$种\\


\textbf{vi. }一对\\
13种数字,每对从4张选2张,剩下三张单牌有48,44,40种可能,除以三张牌的排列$3!$\\
总的来说,$13\times {4\choose 2}\times (48\times 44 \times 40 )/3!$种\\

~\\
\noindent \textbf{122}

\textbf{i. }走17个街区,向东北走,9次向东,8次向北,每次只有两种选择,向东或者向北,所以有
${17 \choose 9} = {17 \choose 8}$种走法

\textbf{ii. }经过水路的数量
$${7\choose 4}{10\choose 5}+{7 \choose 3}{9\choose 4}+{7\choose 2}{9 \choose 4}$$

用i. 的结果减去上式就是结果
\end{document}